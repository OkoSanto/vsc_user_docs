\chapter{Configure access and security for instances}

Before you launch an \gls{instance}, you should add security group rules to
enable users to ping and use SSH to connect to the instance. Security
groups are sets of IP filter rules that define networking access and are
applied to all instances within a project. To do so, you either add
rules to the default security group
{\emph{Add a rule to the default security group}\ref{add-a-rule-to-the-default-security-group}} or add a new security group with
rules.

Key pairs are SSH credentials that are injected into an instance when it
is launched. To use key pair injection, the image that the instance is
based on must contain the \textbf{cloud-init} package. Each project
should have at least one key pair. For more information, see the section
{\emph{Add a key pair}}.

If you have generated a key pair with an external tool, you can import
it into \gls{OpenStack}. The key pair can be used for multiple instances that
belong to a project. For more information, see the section
{\emph{Import a key pair}}.

~

Note

A key pair belongs to an individual user, not to a project. To share a
key pair across multiple users, each user needs to import that key pair.

When an instance is created in \gls{OpenStack}, it is automatically assigned a
fixed IP address in the network to which the instance is assigned. This
IP address is permanently associated with the instance until the
instance is terminated. However, in addition to the fixed IP address, a
floating IP address can also be attached to an instance. Unlike fixed IP
addresses, floating IP addresses are able to have their associations
modified at any time, regardless of the state of the instances involved.

\strong{Add a rule to the default security group}\label{add-a-rule-to-the-default-security-group}

This procedure enables SSH and ICMP (ping) access to instances. The
rules apply to all instances within a given project, and should be set
for every project unless there is a reason to prohibit SSH or ICMP
access to the instances.

This procedure can be adjusted as necessary to add additional security
group rules to a project, if your cloud requires them.

~

Note

When adding a rule, you must specify the protocol used with the
destination port or source port.

\begin{enumerate}
\def\labelenumi{\arabic{enumi}.}
\item
  \begin{quote}
  Log in to the \gls{OpenStack Dashboard}.
  \end{quote}
\item
  \begin{quote}
  Select the appropriate project from the drop down menu at the top
  left.
  \end{quote}
\item
  \begin{quote}
  On the Project tab, open the Network tab. The Security Groups tab
  shows the security groups that are available for this project.
  \end{quote}
\item
  \begin{quote}
  Select the default security group and click Manage Rules.
  \end{quote}
\item
  \begin{quote}
  To allow SSH access, click Add Rule.
  \end{quote}
\item
  \begin{quote}
  In the Add Rule dialog box, enter the following values:
  \end{quote}

  \begin{itemize}
  \item
    \begin{quote}
    \textbf{Rule}: \textbf{SSH}
    \end{quote}
  \item
    \begin{quote}
    \textbf{Remote}: \textbf{CIDR}
    \end{quote}
  \item
    \begin{quote}
    \textbf{CIDR}: \textbf{0.0.0.0/0}
    \end{quote}
  \end{itemize}
\end{enumerate}

\begin{quote}
~

Note

To accept requests from a particular range of IP addresses, specify the
IP address block in the CIDR box.
\end{quote}

\begin{enumerate}
\def\labelenumi{\arabic{enumi}.}
\item
  \begin{quote}
  Click Add.
  \end{quote}
\end{enumerate}

\begin{quote}
Instances will now have SSH port 22 open for requests from any IP address.
\end{quote}

\begin{enumerate}
\def\labelenumi{\arabic{enumi}.}
\item
  \begin{quote}
  To add an ICMP rule, click Add Rule.
  \end{quote}
\item
  \begin{quote}
  In the Add Rule dialog box, enter the following values:
  \end{quote}

  \begin{itemize}
  \item
    \begin{quote}
    \textbf{Rule}: \textbf{All ICMP}
    \end{quote}
  \item
    \begin{quote}
    \textbf{Direction}: \textbf{Ingress}
    \end{quote}
  \item
    \begin{quote}
    \textbf{Remote}: \textbf{CIDR}
    \end{quote}
  \item
    \begin{quote}
    \textbf{CIDR}: \textbf{0.0.0.0/0}
    \end{quote}
  \end{itemize}
\item
  \begin{quote}
  Click Add.
  \end{quote}
\end{enumerate}

\begin{quote}
Instances will now accept all incoming ICMP packets.
\end{quote}

\strong{Add a key pair}\label{add-a-key-pair}

Create at least one key pair for each project.

\begin{enumerate}
\def\labelenumi{\arabic{enumi}.}
\item
  \begin{quote}
  Log in to the \gls{OpenStack Dashboard}.
  \end{quote}
\item
  \begin{quote}
  Select the appropriate project from the drop down menu at the top
  left.
  \end{quote}
\item
  \begin{quote}
  On the Project tab, open the Compute tab.
  \end{quote}
\item
  \begin{quote}
  Click the Key Pairs tab, which shows the key pairs that are available
  for this project.
  \end{quote}
\item
  \begin{quote}
  Click Create Key Pair.
  \end{quote}
\item
  \begin{quote}
  In the Create Key Pair dialog box, enter a name for your key pair, and
  click Create Key Pair.
  \end{quote}
\item
  \begin{quote}
  Respond to the prompt to download the key pair.
  \end{quote}
\end{enumerate}

\strong{Import a key pair}\label{import-a-key-pair}

\begin{enumerate}
\def\labelenumi{\arabic{enumi}.}
\item
  \begin{quote}
  Log in to the \gls{OpenStack Dashboard}.
  \end{quote}
\item
  \begin{quote}
  Select the appropriate project from the drop down menu at the top
  left.
  \end{quote}
\item
  \begin{quote}
  On the Project tab, open the Compute tab.
  \end{quote}
\item
  \begin{quote}
  Click the Key Pairs tab, which shows the key pairs that are available
  for this project.
  \end{quote}
\item
  \begin{quote}
  Click Import Key Pair.
  \end{quote}
\item
  \begin{quote}
  In the Import Key Pair dialog box, enter the name of your key pair,
  copy the public key into the Public Key box, and then click Import Key
  Pair.
  \end{quote}
\item
  \begin{quote}
  Save the \textbf{*.pem} file locally.
  \end{quote}
\item
  \begin{quote}
  To change its permissions so that only you can read and write to the
  file, run the following command:
  \end{quote}
\item
  \begin{quote}
  \$ chmod 0600 yourPrivateKey.pem
  \end{quote}
\end{enumerate}

\begin{quote}
~

Note

If you are using the \gls{OpenStack Dashboard} from a Windows computer, use PuTTYgen to
load the \textbf{*.pem} file and convert and save it as \textbf{*.ppk}.
For more information see the
\href{https://winscp.net/eng/docs/ui_puttygen}{\emph{WinSCP web page for
PuTTYgen}}.
\end{quote}

\begin{enumerate}
\def\labelenumi{\arabic{enumi}.}
\item
  \begin{quote}
  To make the key pair known to SSH, run the \textbf{ssh-add} command.
  \end{quote}
\item
  \begin{quote}
  \$ ssh-add yourPrivateKey.pem
  \end{quote}
\end{enumerate}

The Compute database registers the public key of the key pair.

The \gls{OpenStack Dashboard} lists the key pair on the Key Pairs tab.

\strong{Allocate a floating IP address to an instance}\label{allocate-a-floating-ip-address-to-an-instance}

When an instance is created in \gls{OpenStack}, it is automatically assigned a
fixed IP address in the network to which the instance is assigned. This
IP address is permanently associated with the instance until the
instance is terminated.

However, in addition to the fixed IP address, a floating IP address can
also be attached to an instance. Unlike fixed IP addresses, floating IP
addresses can have their associations modified at any time, regardless
of the state of the instances involved. This procedure details the
reservation of a floating IP address from an existing pool of addresses
and the association of that address with a specific instance.

\begin{enumerate}
\def\labelenumi{\arabic{enumi}.}
\item
  \begin{quote}
  Log in to the \gls{OpenStack Dashboard}.
  \end{quote}
\item
  \begin{quote}
  Select the appropriate project from the drop down menu at the top
  left.
  \end{quote}
\item
  \begin{quote}
  On the Project tab, open the Network tab.
  \end{quote}
\item
  \begin{quote}
  Click the Floating IPs tab, which shows the floating IP addresses
  allocated to instances.
  \end{quote}
\item
  \begin{quote}
  Click Allocate IP To Project.
  \end{quote}
\item
  \begin{quote}
  Choose the pool from which to pick the IP address.
  \end{quote}
\item
  \begin{quote}
  Click Allocate IP.
  \end{quote}
\item
  \begin{quote}
  In the Floating IPs list, click Associate.
  \end{quote}
\item
  \begin{quote}
  In the Manage Floating IP Associations dialog box, choose the
  following options:
  \end{quote}

  \begin{itemize}
  \item
    \begin{quote}
    The IP Address field is filled automatically, but you can add a new
    IP address by clicking the + button.
    \end{quote}
  \item
    \begin{quote}
    In the Port to be associated field, select a port from the list.
    \end{quote}
  \end{itemize}
\end{enumerate}

\begin{quote}
The list shows all the instances with their fixed IP addresses.
\end{quote}

\begin{enumerate}
\def\labelenumi{\arabic{enumi}.}
\item
  \begin{quote}
  Click Associate.
  \end{quote}
\end{enumerate}


Another way to associate a floating IP is after the user has already launched an instance which appears in the list of running instances in the Project-->Compute-->Instances tab:

\begin{enumerate}
\def\labelenumi{\arabic{enumi}.}
\item
  \begin{quote}
  Expand the drop-down menu on right next to the instance
  \end{quote}
\item
  \begin{quote}
  Select Associate Floating IP
\begin{center}
\includegraphics[scale=0.7]{img/associate_IP_1.png}
\end{center}
  \end{quote}
\item
  \begin{quote}
  A pop-up window will appear and under IP Address select from the drop-down menu an IP address from the available pool.
\begin{center}
\includegraphics[scale=0.5]{img/associate_IP_2.png}
\end{center}
  \end{quote}
\item
  \begin{quote}
  Click Associate
  \end{quote}
\end{enumerate}

If the IP has been successfully associated in the upper right corner of the browser screen will appear a green confirmation. If not successful a red notification will pop up that something went wrong.

~

Note

To disassociate an IP address from an instance, click the Disassociate
button.

To release the floating IP address back into the floating IP pool, click
the Release Floating IP option in the Actions column.
