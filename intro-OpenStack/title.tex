\begin{center}

\includegraphics*[width=12truecm]{logo_vsc_2018}

\vspace*{6\baselineskip}

\Huge \strong{Linux Tutorial} \\
\LARGE Last updated: DATEPLACEHOLDER

\LARGE For \OS Users

\vspace*{3\baselineskip}

\normalsize\strong{Authors:}

Alexander Vapirev (KU Leuven), Thomas Danckaert (UAntwerpen)

\vspace*{.5\baselineskip}

Acknowledgement: VSCentrum.be

\vfill

\begin{tabular}{ >{\centering\arraybackslash}m{0.25\textwidth}  >{\centering\arraybackslash}m{0.05\textwidth}  >{\centering\arraybackslash}m{0.20\textwidth}  >{\centering\arraybackslash}m{0.2\textwidth}} \\
\includegraphics*[width=0.2\textwidth, height=0.7in, keepaspectratio=true]{logo_auha} & \multicolumn{2}{ >{\centering\arraybackslash}m{.2\textwidth} }{\includegraphics*[width=0.2\textwidth, height=0.7in,, keepaspectratio=true]{logo_akuleuven}} & \includegraphics*[width=0.2\textwidth, height=0.7in,, keepaspectratio=true]{logo_auhl} \\
\multicolumn{2}{ >{\centering\arraybackslash}m{.32\textwidth} }{\includegraphics*[width=0.3\textwidth, height=0.7in, keepaspectratio=true]{logo_augent}} & \multicolumn{2}{ >{\centering\arraybackslash}m{.38\textwidth} }{\includegraphics*[width=0.3\textwidth, height=0.7in, keepaspectratio=false]{logo_uab}} \\
\end{tabular}
\end{center}

\cleardoublepage
\pagestyle{plain}
\strong{Audience:}

This document is a hands-on guide for using the \gls{OpenStack} Cloud Computing platform in the
context of the \strong{\university} \hpc Tier1 infrastructure. The resources are mostly deployed as infrastructure-as-a-service (IaaS).
These resources can be managed either via a web-based UI (\gls{OpenStack Dashboard}), or using command line and scripts via the OpenStack API.

\strong{\underbar{Notification:}}



``Text'' Is the notation for text to be entered.

\begin{tip}
A ``Tip'' paragraph is used for remarks or tips.
\end{tip}

They can also be downloaded from the VSC website at
\url{https://www.vscentrum.be}.
Apart from this \hpc Tutorial, the documentation on the VSC website
will serve as a reference for all the
operations.


\begin{tip}
The users are advised to get self-organised. There are
only limited resources available at the \hpc, which are best effort based.
The \hpc cannot give support for code fixing, the user applications and own
developed software remain solely the responsibility of the end-user.
\end{tip}

More documentation can be found at:

\begin{enumerate}
  \item  VSC documentation: \url{https://www.vscentrum.be/en/user-portal}
  \ifantwerpen
    \item CalcUA Core Facility web pages: \url{https://www.uantwerpen.be/hpc}
  \fi
  \ifbrussel
    \item \hpcname documentation: \url{http://cc.ulb.ac.be/hpc}
  \fi
  \item  External documentation (OpenStack): \url{https://docs.openstack.org}
\end{enumerate}

This tutorial is intended for users who want to use the cloud services offered within the HPC infrastrucutre of the \strong{\university}.

This tutorial is available in a Windows, Mac or Linux version.

This tutorial is available for UAntwerpen, UGent, KU~Leuven, UHasselt and VUB users.

Request your appropriate version at \hpcinfo.

\strong{\underbar{Contact Information:}}

We welcome your feedback, comments and suggestions for improving the OpenStack
Tutorial (contact: \cloudinfo).

For all technical questions, please contact the \hpc staff:

\inputsite{contact-information}
