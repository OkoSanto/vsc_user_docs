\begin{center}

\includegraphics*[width=12truecm]{logo_vsc_2018}

\vspace*{6\baselineskip}

\Huge \strong{OpenStack Tutorial}\\
\LARGE Last updated: DATEPLACEHOLDER

\LARGE For \OS Users

\vspace*{3\baselineskip}

\normalsize\strong{Authors:}

Alexander Vapirev (KU Leuven), Thomas Danckaert (UAntwerpen)

\vspace*{.5\baselineskip}

Acknowledgement: VSCentrum.be

\vfill

\begin{tabular}{ >{\centering\arraybackslash}m{0.25\textwidth}  >{\centering\arraybackslash}m{0.05\textwidth}  >{\centering\arraybackslash}m{0.20\textwidth}  >{\centering\arraybackslash}m{0.2\textwidth}} \\
\includegraphics*[width=0.2\textwidth, height=0.7in, keepaspectratio=true]{logo_auha} & \multicolumn{2}{ >{\centering\arraybackslash}m{.2\textwidth} }{\includegraphics*[width=0.2\textwidth, height=0.7in,, keepaspectratio=true]{logo_akuleuven}} & \includegraphics*[width=0.2\textwidth, height=0.7in,, keepaspectratio=true]{logo_auhl} \\
\multicolumn{2}{ >{\centering\arraybackslash}m{.32\textwidth} }{\includegraphics*[width=0.3\textwidth, height=0.7in, keepaspectratio=true]{logo_augent}} & \multicolumn{2}{ >{\centering\arraybackslash}m{.38\textwidth} }{\includegraphics*[width=0.3\textwidth, height=0.7in, keepaspectratio=false]{logo_uab}} \\
\end{tabular}
\end{center}

\cleardoublepage
\pagestyle{plain}
\strong{Audience:}

This document is a hands-on guide for using the VSC \gls{OpenStack}
cloud computing platform.  It should complement the official documentation at
\url{https://docs.openstack.org}.


\strong{\underbar{Notification:}}

``Text'' Is the notation for text to be entered.

\begin{tip}
A ``Tip'' paragraph is used for remarks or tips.
\end{tip}

This tutorial is available in a Windows, Mac or Linux version.

This tutorial is available for UAntwerpen, UGent, KU~Leuven, UHasselt and VUB users.

\strong{\underbar{Contact Information:}}


For all questions concerning the VSC cloud computing platform, please contact \cloudinfo.

We welcome your feedback, comments and suggestions for improving the OpenStack Tutorial

%%% Local Variables:
%%% mode: latex
%%% TeX-master: "intro-OpenStack"
%%% End:
