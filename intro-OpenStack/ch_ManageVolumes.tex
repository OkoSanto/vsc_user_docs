\chapter{Create and manage volumes}

An \gls{OpenStack Volume} is a block storage devices which you attach to instances to enable persistent storage. You can attach a volume to a running instance or
detach a volume and attach it to another instance at any time. You can
also create a snapshot from or delete a volume. Only administrative
users can create volume types.

\strong{Create a volume}\label{create-a-volume}

\begin{enumerate}
\def\labelenumi{\arabic{enumi}.}
\item
  \begin{quote}
  Log in to the dashboard.
  \end{quote}
\item
  \begin{quote}
  Select the appropriate project from the drop down menu at the top
  left.
  \end{quote}
\item
  \begin{quote}
  On the Project tab, open the Volumes tab and click Volumes category.
  \end{quote}
\item
  \begin{quote}
  Click Create Volume.
  \end{quote}
\end{enumerate}

\begin{quote}
In the dialog box that opens, enter or select the following values.

Volume Name: Specify a name for the volume.

Description: Optionally, provide a brief description for the volume.

Volume Source: Select one of the following options:
\end{quote}

\begin{itemize}
\item
  \begin{quote}
  No source, empty volume: Creates an empty volume. An empty volume does
  not contain a file system or a partition table.
  \end{quote}
\item
  \begin{quote}
  Snapshot: If you choose this option, a new field for Use snapshot as a
  source displays. You can select the snapshot from the list.
  \end{quote}
\item
  \begin{quote}
  Image: If you choose this option, a new field for Use image as a
  source displays. You can select the image from the list.
  \end{quote}
\item
  \begin{quote}
  Volume: If you choose this option, a new field for Use volume as a
  source displays. You can select the volume from the list. Options to
  use a snapshot or a volume as the source for a volume are displayed
  only if there are existing snapshots or volumes.
  \end{quote}
\end{itemize}

\begin{quote}
Type: Leave this field blank.

Size (GB): The size of the volume in gibibytes (GiB).

Availability Zone: Select the Availability Zone from the list. By
default, this value is set to the availability zone given by the cloud
provider (for example, \textbf{us-west} or \textbf{apac-south}). For
some cases, it could be \textbf{nova}.
\end{quote}

\begin{enumerate}
\def\labelenumi{\arabic{enumi}.}
\item
  \begin{quote}
  Click Create Volume.
  \end{quote}
\end{enumerate}

The dashboard shows the volume on the Volumes tab.

\strong{Attach a volume to an instance}\label{attach-a-volume-to-an-instance}

After you create one or more volumes, you can attach them to instances.
You can attach a volume to one instance at a time.

\begin{enumerate}
\def\labelenumi{\arabic{enumi}.}
\item
  \begin{quote}
  Log in to the dashboard.
  \end{quote}
\item
  \begin{quote}
  Select the appropriate project from the drop down menu at the top
  left.
  \end{quote}
\item
  \begin{quote}
  On the Project tab, open the Volumes tab and click Volumes category.
  \end{quote}
\item
  \begin{quote}
  Select the volume to add to an instance and click Manage Attachments.
  \end{quote}
\item
  \begin{quote}
  In the Manage Volume Attachments dialog box, select an instance.
  \end{quote}
\item
  \begin{quote}
  Enter the name of the device from which the volume is accessible by
  the instance.
  \end{quote}
\end{enumerate}

\begin{quote}
~

Note

The actual device name might differ from the volume name because of
hypervisor settings.
\end{quote}

\begin{enumerate}
\def\labelenumi{\arabic{enumi}.}
\item
  \begin{quote}
  Click Attach Volume.
  \end{quote}
\end{enumerate}

\begin{quote}
The dashboard shows the instance to which the volume is now attached and
the device name.
\end{quote}

You can view the status of a volume in the Volumes tab of the dashboard.
The volume is either Available or In-Use.

Now you can log in to the instance and mount, format, and use the disk.

\strong{Detach a volume from an instance}\label{detach-a-volume-from-an-instance}

\begin{enumerate}
\def\labelenumi{\arabic{enumi}.}
\item
  \begin{quote}
  Log in to the dashboard.
  \end{quote}
\item
  \begin{quote}
  Select the appropriate project from the drop down menu at the top
  left.
  \end{quote}
\item
  \begin{quote}
  On the Project tab, open the Volumes tab and click the Volumes
  category.
  \end{quote}
\item
  \begin{quote}
  Select the volume and click Manage Attachments.
  \end{quote}
\item
  \begin{quote}
  Click Detach Volume and confirm your changes.
  \end{quote}
\end{enumerate}

A message indicates whether the action was successful.

\strong{Create a snapshot from a volume}\label{create-a-snapshot-from-a-volume}

\begin{enumerate}
\def\labelenumi{\arabic{enumi}.}
\item
  \begin{quote}
  Log in to the dashboard.
  \end{quote}
\item
  \begin{quote}
  Select the appropriate project from the drop down menu at the top
  left.
  \end{quote}
\item
  \begin{quote}
  On the Project tab, open the Volumes tab and click Volumes category.
  \end{quote}
\item
  \begin{quote}
  Select a volume from which to create a snapshot.
  \end{quote}
\item
  \begin{quote}
  In the Actions column, click Create Snapshot.
  \end{quote}
\item
  \begin{quote}
  In the dialog box that opens, enter a snapshot name and a brief
  description.
  \end{quote}
\item
  \begin{quote}
  Confirm your changes.
  \end{quote}
\end{enumerate}

\begin{quote}
The dashboard shows the new volume snapshot in Volume Snapshots tab.
\end{quote}

\strong{Edit a volume}\label{edit-a-volume}

\begin{enumerate}
\def\labelenumi{\arabic{enumi}.}
\item
  \begin{quote}
  Log in to the dashboard.
  \end{quote}
\item
  \begin{quote}
  Select the appropriate project from the drop down menu at the top
  left.
  \end{quote}
\item
  \begin{quote}
  On the Project tab, open the Volumes tab and click Volumes category.
  \end{quote}
\item
  \begin{quote}
  Select the volume that you want to edit.
  \end{quote}
\item
  \begin{quote}
  In the Actions column, click Edit Volume.
  \end{quote}
\item
  \begin{quote}
  In the Edit Volume dialog box, update the name and description of the
  volume.
  \end{quote}
\item
  \begin{quote}
  Click Edit Volume.
  \end{quote}
\end{enumerate}

\begin{quote}
~

Note

You can extend a volume by using the Extend Volume option available in
the More dropdown list and entering the new value for volume size.
\end{quote}

\strong{Delete a volume}\label{delete-a-volume}

When you delete an instance, the data in its attached volumes is not
deleted.

\begin{enumerate}
\def\labelenumi{\arabic{enumi}.}
\item
  \begin{quote}
  Log in to the dashboard.
  \end{quote}
\item
  \begin{quote}
  Select the appropriate project from the drop down menu at the top
  left.
  \end{quote}
\item
  \begin{quote}
  On the Project tab, open the Volumes tab and click Volumes category.
  \end{quote}
\item
  \begin{quote}
  Select the check boxes for the volumes that you want to delete.
  \end{quote}
\item
  \begin{quote}
  Click Delete Volumes and confirm your choice.
  \end{quote}
\end{enumerate}

\begin{quote}
A message indicates whether the action was successful.
\end{quote}
